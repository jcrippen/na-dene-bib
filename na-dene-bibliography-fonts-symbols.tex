%!TEX root = ./na-dene-bibliography.tex
%!TEX encoding = UTF-8 Unicode
%%
%% Font support for special symbols (e.g. math).
%%

% https://tex.stackexchange.com/questions/367773/command-textsuperscript-unavailable-in-encoding-tu?noredirect=1
\DeclareTextSymbol{\texttilde} {TU} {"007E}

%% Load the newunicodechar package.
%%
%% This package lets you define arbitrary macros to run for particular
%% Unicode characters (or any character, really). This allows you to
%% switch fonts per character, and thus use different fonts for e.g.
%% mathematical symbols, Hebrew letters, etc. It should not be used
%% for long stretches of text, which instead should be done with a
%% specialized command or environment that changes the font in a
%% more general way.
%%
%% All that the \newunicodechar command is really doing under the
%% hood is something like plain TeX
%%   \catcode`\X=\active
%%   \defX{\textbf{\char"2323}}
%% but it's much easier to use and so less error-prone.
\usepackage{newunicodechar}

%% The Brill font has Greek-style default versions of {β, θ, λ, χ} and
%% Roman-style versions available with ss20. But I only want
%% the roman θ and not the others. Brill 2.02 doesn’t implement
%% character variant features (cvXX) for these, so the only way to access
%% them is through ss20. I use newunicodechar to automatically add
%% ss20 whenever θ appears.
%%
%% This solution completely ignores the current font. That means that ss20 will
%% be applied in every situation, no matter what. So even if the font is no
%% longer Brill for some reason, θ will still show up with ss20. It works fine if
%% you’re consistent and careful, but it isn’t ideal. cvXX would be ideal.
\newunicodechar{θ}{{\addfontfeature{StylisticSet=20}θ}}
